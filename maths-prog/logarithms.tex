\documentclass[12pt,oneside,a4paper]{article}

\pagenumbering{gobble}

\usepackage[margin=2cm]{geometry}
\usepackage{relsize}
\usepackage{exscale}
\usepackage{ amssymb }
\usepackage{amsthm}
\usepackage{amsmath}
\usepackage[bottom]{footmisc}
\usepackage{hyperref}


\newtheorem{property}{Property}
\newtheorem{lemma}{Lemma} 
\begin{document}
\title{Equality of $\displaystyle \mathop{\mathlarger{\mathlarger{\int}}}_{1}^{x} \frac{1}{z} dz 
$ and the inverse of $e^x$
\footnote{Congratulations to the Math With Bad Drawings Blog \url{https://mathwithbaddrawings.com/2018/08/15/the-bubble-under-the-mathematical-rug/} for having nerd-snipped me so hard}} 
\date{} % So that no date appears
\author{Nu\~{n}o Sempere}
\maketitle

% I like my integrals large: \mathop{\mathlarger{\mathlarger{\int}}}
\section{Proofs of some properties of the logarithm}
Let the natural logarithm be defined as:
\begin{equation}
ln(x) := \mathop{\mathlarger{\mathlarger{\int}}}_{1}^{x} \frac{1}{z} dz 
\end{equation}

\begin{property}
$ln(x \cdot y) = ln(x) + ln(y)$
\end{property}
\begin{proof}
Consider $f(x)$ and $g(x)$:

\begin{equation}
f_x(y)= \mathop{\mathlarger{\mathlarger{\int}}}_{1}^{x \cdot y} \frac{1}{z} dz
= ln(x \cdot y)
\end{equation}
\begin{equation}
g_x(y)= \mathop{\mathlarger{\mathlarger{\int}}}_{1}^{x} \frac{1}{z} dz + \mathop{\mathlarger{\mathlarger{\int}}}_{1}^{y} \frac{1}{z} dz
= ln(x) + ln(y)
\end{equation}
differentiating both with respect to $y$, we get that:
\begin{equation}
f_x'(y)= \frac{1}{xy} \cdot (x \cdot y)' = \frac{1}{y}
\end{equation}
\begin{equation}
g_x'(y)= 0+\frac{1}{y} \cdot (y)' = \frac{1}{y}
\end{equation}
thus, 
\begin{equation}
\forall x \; g_x'(y) = f_x'(y) 
\end{equation}
additionally,
\begin{equation}
\forall x \; f_x(1) = g_x(1) = ln(x)
\end{equation}
From this we can conclude, by integration, that $f_x(y) = g_x(y)$, which completes our proof.
\end{proof}
\par \noindent
We will use again and again the trick of differentiating and checking equality for one value to prove that two functions are the same. We leave it as an exercise to the reader to check whether this is legal, i.e., whether we comply with the hypothesis of the Picard-Lindel{\"o}f theorem or similar.
\newpage
\begin{property}
$ln(x ^n) = n \cdot ln(x)$
\end{property}

\begin{proof}
Again, consider $f(x)$ and $g(x)$:
\begin{equation}
f_n(x)= \mathop{\mathlarger{\mathlarger{\int}}}_{1}^{x ^n} \frac{1}{z} dz
= ln(x^n)
\end{equation}
\begin{equation}
g_n(x)= n \cdot \mathop{\mathlarger{\mathlarger{\int}}}_{1}^{x} \frac{1}{z} dz = n\cdot ln(x)
\end{equation}
For clarity, let $F(x)$ be a primitive of $\frac{1}{x}$, so that:
\begin{equation}
f_n(x)= F(x^n) - F(1)
\end{equation}
\begin{equation}
g_n(x)= n\cdot ( F(x) - F(1))
\end{equation}
differentiating both with respect to $x$, we get that:
\begin{equation}
f_n'(x)= F'(x^n) \cdot (x^n)' -0 = F'(x^n) \cdot n \cdot x^{n-1} = \frac{1}{x^n}\cdot n \cdot x^{n-1} = \frac{n}{x}
\end{equation}
\begin{equation}
g_n'(x)= n\cdot ( F'(x) - 0) = \frac{n}{x}
\end{equation}
Again, $\forall n \; f_n'(x)=g_n'(x) = \frac{n}{x} \land f_n(1)=g_n(1) \implies f_n(x) = g_n(x)$
\end{proof}
\section{The exponential function as the inverse of the logarithm}
Let $exp(x)$ be the inverse of the logarithm function, that is:
\begin{equation}
exp(ln(x))=x
\end{equation}
Note that the inverse exists because the logarithm is a strictly increasing function.
\begin{property}
$exp(0)=1$, and the exponential is it's own derivative: $exp'(x)= exp(x)$.
\end{property}
\begin{proof}
For the first part, $exp(ln(1))=1 \implies exp(0)=1$. 
For the second part, write the exponential as:
\begin{equation}
exp\left(\mathop{\mathlarger{\mathlarger{\int}}}_{1}^{x} \frac{1}{z} dz\right)=x
\end{equation}
differentiating the above expression with respect to x:
\begin{equation}
exp'\left(\mathop{\mathlarger{\mathlarger{\int}}}_{1}^{x} \frac{1}{z} dz\right)\cdot \left(\mathop{\mathlarger{\mathlarger{\int}}}_{1}^{x} \frac{1}{z} dz\right)' = exp'\left(\mathop{\mathlarger{\mathlarger{\int}}}_{1}^{x} \frac{1}{z} dz\right)\cdot \frac{1}{x} = 1
\end{equation}
Notice that $(x)' = 1$. Multiplying by $x \neq 0$:
\begin{equation}
exp'\left(\mathop{\mathlarger{\mathlarger{\int}}}_{1}^{x} \frac{1}{z} dz\right) = x
\end{equation}
Note that $log(0)$ is not well defined as an integral, and thus we have no need of an inverse at $x=0$.
At any point, because of the uniqueness of the inverse, and writting $y=ln(x)$, we conclude that $exp'(y) = exp(y)$. Note that $y=ln(x)$, and that the image of the logarithm comprises all real numbers.
\end{proof}
From now on, we would feel justified in using the Taylor expansion of $exp(x)$.
\begin{property}
$e:= \lim_{n \to \infty} \left(1+ \frac{1}{n}\right)^n \implies e^x = \lim_{n \to \infty} \left(1+\frac{x}{n}\right)^n$
\end{property}
\begin{proof}
For a fixed value of $x$,
\begin{equation}
\lim_{n \to \infty} \left(1+\frac{x}{n}\right)^n= \lim_{k = \frac{n}{x} \to \infty} \left(1+\frac{1}{k}\right)^{k \cdot x} = \left(\lim_{k \to \infty} \left(1+\frac{1}{k}\right)^{k} \right)^x = e^x
\end{equation}
\end{proof}
Note the happy coincidence that $e^0 = \lim_{n \to \infty} \left(1+\frac{0}{n}\right)^n = 1 = exp(0)$. Note also that this step simplifies our proof \textit{immensely}, because working with $e^x$ as 
\begin{equation}
lim_{\frac{p}{q} \to x} \sqrt[q]{e^p}	
\end{equation}
would have been torturous.
\begin{property}
The limit $\lim_{n \to \infty} \left(1+ \frac{1}{n}\right)^n$ is bounded and defines the unique value $e$ such that $ln(e)=1 \iff e = exp(1)$
\end{property}
\begin{proof}
Let us take the logarithm of $e$:
\begin{equation}
ln(e) = ln\left(\lim_{n \to \infty} \left(1+ \frac{1}{n}\right)^n\right)
\end{equation}
We can place the logarithm inside the limit and take out the exponent $n$ as a multiplier:
\begin{equation}
ln(e) = \lim_{n \to \infty} ln\left(\left(1+ \frac{1}{n}\right)^n\right) = \lim_{n \to \infty} n\cdot ln\left(1+ \frac{1}{n}\right)
\end{equation}
\begin{equation}
ln(e) = \lim_{n \to \infty} \frac{ln\left(1+ \frac{1}{n}\right)}{1/n}
\end{equation}
Now, because of L'Hopital's rule, we know that:
\begin{equation}
\lim_{t \to 0} \frac{ln(1+t)}{t} = 1
\end{equation}
As a brief remainder, replace the $ln(t+1)$ by its Taylor expansion, and note that the higher power terms leave $0$s. Thus, defining $t=1/n$, we have:
\begin{equation}
ln(e) = \lim_{t \to 0} \frac{ln\left(1+ t\right)}{t} = 1 \iff e = exp(1)
\end{equation}
It's left as an exercise to the reader to prove that the Taylor expansion of $exp(x)$ is bounded for all $x$, and in particular for $x=1$.
\end{proof}
\newpage
\begin{property}
$e^x = exp(x)$
\end{property}
\begin{proof}\footnote{The proof idea is taken from  \url{https://proofwiki.org/wiki/Exponential_as_Limit_of_Sequence} } Much like above, let us take the logarithm of $e^x$, for a fixed x:
\begin{equation}
ln(e^x) = ln\left( \lim_{n \to \infty} \left(1+\frac{x}{n}\right)^n \right)
\end{equation}
We can place the logarithm inside the limit and then take the exponent $n$ out: 
\begin{equation}
ln(e^x) = \lim_{n \to \infty}ln\left(  \left(1+\frac{x}{n}\right)^n \right)
\end{equation}
\begin{equation}
ln(e^x) = \lim_{n \to \infty}n \cdot ln\left(1+\frac{x}{n}\right)
\end{equation}
We multiply and divide by $\frac{x}{n}$
\begin{equation}
ln(e^x) = \lim_{n \to \infty} x\cdot \frac{ln\left(1+\frac{x}{n}\right)}{x/n}
\end{equation}
We define $t=x/n$, so as $n \to \infty$, $t \to 0$, and apply L'Hopital's rule.
\begin{equation}
ln(e^x) = x \cdot \lim_{t \to 0}\frac{ln\left(1+t \right)}{t} = x
\end{equation}
Thus, $e^x$ is the inverse of $ln(x)$, and because of the uniqueness of the inverse,
\begin{equation}
e^x = exp(x)
\end{equation}
\end{proof}

\end{document}
